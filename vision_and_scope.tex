\documentclass{article}
\usepackage{url}
\usepackage[spacing,kerning]{microtype}
\usepackage[letterpaper]{geometry}

\title{Vision and Scope Document\\
\bigskip
{\large for}\\
\bigskip
PhysioMIST}
\author{Mark Caral, Sara Cummins, BarbaraJoy Jones, Joshua Lee}
\date{September 14, 2009\\{\sc Eecs} 393}
\begin{document}

\begin{titlepage}
\maketitle
\end{titlepage}

\tableofcontents
\newpage

\section{Requirements}
\subsection{Background}
PhysioMIST is an emerging open source programming interface for modeling human physiology by integrating mathematical models of physiological processes.
This extends the capabilities of popular simulation software such as JSim.
This environment allows for comprehensive analysis of complex biological systems by studying different levels and scales of models (e.g., bodily systems, organs, and tissues) rather than individual components.
One aspect of integrating models is associating human anatomical data with specific model components. This data is represented as an ontological system of modular relationships.
The project uses an information flow interface to decrease coupling between dependent models, thus simplifying an inherently complex process.

The most important feature of the PhysioMIST software is the model integration process.
This project will entail the creation of the model integration interface including the GUI provided to the user as well as the underlying representation of mathematical models and the data passed to the integration and simulation mechanisms.
Implementing this project will allow the PhysioMIST user to extend their research into human biological processes by integrating existing models into their experimental research.
Model integration will lead researchers to build more accurate and complete models for investigating specific physiological phenomena to improve the quality of medical and biological research.

\subsection{Objectives and Success Criteria}
\begin{description}
\item[O-1] Provide a graphical interface for users to associate anatomical information with model components
\item[O-2] Provide a graphical interface for users to integrate components from one or more models on a one-to-one basis
\item[O-3] Provide a graphical interface for users to adopt other types of model integration, such as many-to-one or one-to-many
\item[O-4] Provide a robust representation of the information contained in a model
\item[SC-1] Allow an experienced JSim user to become comfortable working with existing models within 2 days
\item[SC-2] Efficiently provide the user with accurate integration results
\item[SC-3] The responsiveness of the GUI will not be affected by the complexity of the model integration
\end{description}

\subsection{Risks}
\begin{description}
\item[RI-1] The PhysioMIST software may not be adopted by current JSim users
\item[RI-2] Integrated models produced by PhysioMIST may be too complex to be imported by existing software
\end{description}

\section{Vision of the Solution}
\subsection{High Priority Features}
\begin{itemize}
\item Import existing models written in the JSim standard
\item Save anatomical information associated with model components
\item Integrate model components on a one-to-one basis
\end{itemize}
\subsection{Low Priority Features}
\begin{itemize}
\item Integrate model components on a one-to-many basis
\item Integrate model components on a many-to-one basis
\item Display simulation results for integrated models
\end{itemize}
\subsection{Assumptions and Dependencies}
This project makes the assumption that the existing PhysioMIST codebase it depends on is easily extendable and capable of performing the integration and simulation processes.

\section{Scope and Limitations}
\subsection{Release Scope}
Low priority features may not be implemented in the planned December release.
\subsection{Limitations and Exclusions}
The PhysioMIST software only accepts models in the JSim standard format and the PhysioMIST internal format.

\section{See Also}
\url{http://robotics.case.edu/modeling_simulation_biological_systems.html}

\end{document}
